
\documentclass{article}

\usepackage{fullpage,latexsym,picinpar,amsmath,amsfonts, hyperref}



\setlength{\evensidemargin}{0.1in}
\setlength{\oddsidemargin}{0.1in}
\setlength{\textwidth}{6.6in}
\setlength{\topmargin}{0.0in}
\setlength{\textheight}{8.7in}
\setlength{\headheight}{0in}
\setlength{\headsep}{0in}
\setlength{\topsep}{0in}
\setlength{\itemsep}{0in}
\renewcommand{\baselinestretch}{1.1}
\parskip=0.080in

\newcommand{\parend}[1]{{\left( #1  \right) }}
\newcommand{\spparend}[1]{{\left(\, #1  \,\right) }}
\newcommand{\angled}[1]{{\left\langle #1  \right\rangle }}
\newcommand{\brackd}[1]{{\left[ #1  \right] }}
\newcommand{\spbrackd}[1]{{\left[\, #1  \,\right] }}
\newcommand{\braced}[1]{{\left\{ #1  \right\} }}
\newcommand{\leftbraced}[1]{{\left\{ #1  \right. }}
\newcommand{\floor}[1]{{\left\lfloor #1\right\rfloor}}
\newcommand{\ceiling}[1]{{\left\lceil #1\right\rceil}}
\newcommand{\barred}[1]{{\left|#1\right|}}
\newcommand{\doublebarred}[1]{{\left|\left|#1\right|\right|}}
\newcommand{\spaced}[1]{{\, #1\, }}
\newcommand{\suchthat}{{\spaced{|}}}
\newcommand{\numof}{{\sharp}}

\newcommand{\half}{{\textstyle\frac{1}{2}}}
\newcommand{\elevenhalves}{{\textstyle\frac{11}{2}}}
\newcommand{\onethird}{{\textstyle\frac{1}{3}}}
\newcommand{\sixteenthirds}{{\textstyle\frac{16}{3}}}
\newcommand{\twentytwothirds}{{\textstyle\frac{22}{3}}}
\newcommand{\onefifth}{{\textstyle\frac{1}{5}}}
\newcommand{\threefifths}{{\textstyle\frac{3}{5}}}
\newcommand{\sixfifths}{{\textstyle\frac{6}{5}}}
\newcommand{\eightfifths}{{\textstyle\frac{8}{5}}}
\newcommand{\sixteenfifths}{{\textstyle\frac{16}{5}}}
\newcommand{\eightteenfifths}{{\textstyle\frac{18}{5}}}
\newcommand{\threetenths}{{\textstyle\frac{3}{10}}}
\newcommand{\twentysixfifteenths}{{\textstyle\frac{26}{15}}}
\newcommand{\fisefiftieths}{{\textstyle\frac{57}{50}}}
\newcommand{\ftwotfifths}{{\textstyle\frac{42}{25}}}
\newcommand{\fotwontwfifths}{{\textstyle\frac{42}{125}}}
\newcommand{\eithontwfifths}{{\textstyle\frac{83}{125}}}

\newcommand{\veps}{{\varepsilon}}
\newcommand{\Sigmastar}{{\Sigma^\ast}}

\newcounter{exnum}[section]
\newenvironment{problem}{{\vskip 0.1in
   \noindent \bf Problem\addtocounter{exnum}{1}~\arabic{exnum}.}}{\vskip 0.1in}

\newtheorem{theorem}{Theorem}
\newtheorem{definition}{Definition}
\newtheorem{corollary}{Corollary}
\newtheorem{lemma}{Lemma}
\newtheorem{fact}{Fact}
\newtheorem{claim}{Claim}

\newenvironment{proof}{{\it Proof:\/}}{$\Box$\vskip 0.1in}

\newcommand{\emparagraph}[1]{{\smallskip\noindent{\em #1\/}}}

\newcommand{\assign}{{\,\gets\,}}



\begin{document}

\centerline{\large \bf CS111 Fall'24 ASSIGNMENT 2}

\vskip 0.2in
%%%%%%%%%%%%%%%%%%%%%%%%%%%%

\begin{problem}

\vspace{0.05in}
\noindent  Prove the following statement: 


\noindent 
If $p > 5$ is a prime integer,
then $(p^2 +59)(p^2 -4)\equiv 0 \pmod{60}$.

\vspace{0.1in}
\noindent  
To receive full credit, you must use the method based on the following property of integers: 

\noindent The product of any $k$ consecutive integers is divisible by $k$.

\vspace{0.1in}
\noindent For small extra credit, you can also show how this statement can be proved using the method of arguing by cases.

\vspace{0.1in}
\noindent In order to prove the statement above using the property of integers, we must do the following:

\vspace{0.1in}
-  Express 60 in terms of Prime Factors: 60 = $2^2 * 3 * 5$

\vspace{0.1in}
- Show that $(p^2+59)(p^2-4)$ is divisible by 2, 3, and 5: 

\vspace{0.1in}
\indent\indent - Since p is an odd prime because 2 is the only even prime and $p > 5$, we have $p^2 \equiv 1 \pmod{2}$. Now, $p^2 + 59 \equiv 1 + 1 \equiv 0 \pmod{2}$ and $p^2 - 4 \equiv 1 - 1 \equiv 0 \pmod{2}$. Since we see that both terms are divisible by 2, their product is also divisible by $2^2$ = 4.

\vspace{0.1in}
\indent\indent - Next, for any $p > 3$, $p$ must be congruent to either 1 or 2 $\pmod{3}$ since 3 is prime and isn't divisible by 3. Since $p^2 \equiv 1 \pmod{3}$, $p^2 +59 \equiv 1 + 2 \equiv 0 \pmod{3}$ and $p^2 - 4 \equiv 1 - 1 \equiv 0 \pmod{3}$. If $p \equiv 2 \pmod{3}$ then $p^2 \equiv 4 \equiv 1 \pmod{3}$, proving that both equations hold true.

\vspace{0.1in}
\indent\indent - Lastly, for any $p > 5$, $p$ is congruent to 1,2,3, or 4 $\pmod{5}$. Since $p^2 \equiv 1 \pmod{5}$, it leads for $p^2 + 59 \equiv 1 + 59 \equiv 0 \pmod{5}$ and $p^2 - 4 \equiv 4 - 4 \equiv 0 \pmod{5}$, leading to the product being $0 \pmod{5}$. For $p \equiv 2 \pmod{5}$, we have $p^2 \equiv 4 \pmod{5}$ which allows for $p^2 + 59 \equiv 4 + 59 \equiv 3 \pmod{5}$ and $p^2 - 4 \equiv 4 - 4 \equiv 0 \pmod{5}$, leading to a product of $0 \pmod{5}$. For $p \equiv 3 \pmod{5}$, we have $p^2 \equiv 9 \equiv 4 \pmod{5}$, which actually reduces to to our previous case. And, for $p \equiv 4 \pmod{5}$, we have $p^2 \equiv 16 \equiv 1$, which also reduces to our $p \equiv 1 \pmod{5}$ case, leading to conclude that one of our factors is divisible by 5.

\vspace{0.1in}
- Since we have proven that $(p^2 +59)(p^2 -4)$ is divisible by 4,3, and 5, it is divisible by 60 = $2^2 * 3 * 5$, which also proves that $(p^2 +59)(p^2 -4)\equiv 0 \pmod{60}$.
\end{problem}


%%%%%%%%%%%%%%%%%%%%%%%%%%%%
\vspace{0.05in}
\begin{problem}

\vspace{0.05in}
\noindent  Alice's RSA public key is $P = (e,n) = (7, 4453)$.
Bob sends Alice the message by encoding it as follows.
First he assigns numbers to characters:
A is 7, B is 8, ..., Z is 32, a blank is 33, quotation marks: 34, a coma:  35, a period: 36, an apostrophe: 37. Then he uses RSA to encode each number separately. 

\vspace{0.1in}
\noindent Bob's encoded message is:

\begin{verbatim}
 1400 2218   99 2088 4191   84  843   99 4191 3780  764 4191 2979 2269   99  764 
 2218 2269 2088  843 3015   99 2970 1443 1655   99 3237 2979   99  447 1443 3237 
 1032 2382  871  843 1655   99  871 1443   99 4242  843   99 4191 2269   99  843 
 4191 2269 2979   99  871 1443   99 2382 2269  843   99 4191 2269   99 3237 2979 
   99  871  843 3780  843 1032 2088 1443 2962  843 2916   99 3237 2979   99  764 
 2218 2269 2088   99 2088 4191 2269   99  447 1443 3237  843   99  871 1655 2382 
  843   99 4242  843  447 4191 2382 2269  843   99 2218   99  447 4191 2962   99
 2962 1443   99 3780 1443 2962 1294  843 1655   99 2970 2218 1294 2382 1655  843 
   99 1443 2382  871   99 2088 1443  764   99  871 1443   99 2382 2269  843   99 
 3237 2979   99  871  843 3780  843 1032 2088 1443 2962  843 2916 1400

\end{verbatim}

\vspace{-0.1in}
\noindent Decode Bob's message.
Notice that you only know Alice's public key, but don't know the private key. So you
need to "break" RSA to decrypt Bob's message. For the solution, you need to provide the following:
%
\begin{description}

	\item (a) Describe step by step how you 
	arrived at the solution: show how to find $p$ and $q$, $\phi(n)$ and $d$. 
	\item (b) Show your work for one integer in the message (C = 2218):  the expression, the decrypted integer, the character that it is mapped to.  
    \item (c) To decode the remaining numbers, you need to write a program in C++ (see below) and test it in Gradescope.
	\item (d) Give the decoded message (in integers).
	\item (e) Give Bob's message in
	plaintext. What does it mean and who said it?

\end{description}

\vspace{0.2in}
\noindent For part (c). Your program should :
\begin{description}

	\item (i) Take three integers, $e$, $n$ (the public key for RSA), and $m$ (the number of characters in the message) as input to your program. Next, input the ciphertext.
	\item (ii) Test whether the public key is valid. If not, output a single line ``Public key is not valid!'' and quit the program.
	\item (iv) If the public key is valid, decode the message.
	\item (v) Output $p$ and $q$, $\phi(n)$ and $d$.
	\item (vi) On a new line, output the decoded message in integers.
	\item (vii) On a new line, output the decoded message in English. The characters should be all uppercase. You can assume that the numbers will be assigned to characters according to the mapping above.
 
\end{description}

\noindent More information and specifications will be provided separately.

\vspace{0.2in}
\noindent Upload your code to Gradescope to test. There will be 15-16 (open and hidden) test cases.  Your score for the RSA code will be based on the score that you received in Gradescope. If you have any questions, post them on Slack.


\end{problem}
\vspace{0.2in}

%%%%%%%%%%%%%%%%%%%%%%%%%%%%


\begin{problem}

\medskip\noindent
(a) Compute $5^{1257}\pmod{12}$. Show your work.

\vspace{0.1in}
\indent - Using modular exponentiation, we find that $5^{1257} \equiv 5^1 \equiv 5 \pmod{12}$. This is true because the power of 5 $\pmod{12}$ cycles as shown and it repeats every 2 powers. Computing for 1257, this proves true. 

\medskip\noindent
(b) Compute $8^{-1}\pmod{17}$ by listing the multiples. Show your work.

\vspace{0.1in}
\indent - We solve this by rewriting the equation to $8x \equiv 1 \pmod{17}$ finding that $x = 15$ since 8 * 5 = 120 $\equiv 1 \pmod{17}$

\medskip\noindent
(c) Compute $8^{-1}\pmod{17}$ using Fermat's Little Theorem. Show your work.

\vspace{0.1in}
\indent - By Fermat's Little Theorem, we compute $8^{15} \equiv 15 \pmod{17}$, leading to prove that $x = 15$.

\medskip\noindent
(d) Compute $8^{-11}\pmod{17}$ using Fermat's Little Theorem. Show your work.

\vspace{0.1in}
\indent - By Fermat's Little Theorem, we find that $8^6 = 262144 \equiv 2 \pmod{17}$, leading to $8^{-11} \equiv 2 \pmod{17}$.

\medskip\noindent
(e) Find an integer $x$, $0\le x \le 40$, that satisfies the following congruence:
$-9x + 14 \equiv 16 \pmod{41}$. Show your work. You should not use brute force approach.

\vspace{0.1in}
\indent - After rearranging the equation, we find $-9x \equiv 2 \pmod{41}$ and finding the modular inverse of $-9 \pmod{41}$ which is 29, leading to $x \equiv 2 * 29 \equiv 58 \equiv 34 \pmod{41}$

\end{problem}



\paragraph{Academic integrity declaration.}
The homework papers must include at the end an academic integrity declaration. This should be a brief paragraph where you state
\emph{in your own words}  (1) whether you did the homework individually or in collaboration with a partner student (if so, provide the name), 
and (2) whether you used any external help or resources. 


%%%%%%%%%%%%%%%%%%%%%%%%%%%%

\vskip 0.1in
\paragraph{Submission.}
To submit the homework, you need to upload the pdf and cpp files to Gradescope. If you submit with a partner, you need
to put two names on the assignment and submit it as a group assignment.

\paragraph{Reminders.}
Remember that only {\LaTeX} papers are accepted. 

\end{document} 

