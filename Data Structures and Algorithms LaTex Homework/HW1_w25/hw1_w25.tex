
\documentclass{article}

\usepackage{fullpage,latexsym,picinpar,amsmath,amsfonts}



\setlength{\evensidemargin}{0.1in}
\setlength{\oddsidemargin}{0.1in}
\setlength{\textwidth}{6.6in}
\setlength{\topmargin}{0.0in}
\setlength{\textheight}{8.7in}
\setlength{\headheight}{0in}
\setlength{\headsep}{0in}
\setlength{\topsep}{0in}
\setlength{\itemsep}{0in}
\renewcommand{\baselinestretch}{1.1}
\parskip=0.080in

\newcommand{\parend}[1]{{\left( #1  \right) }}
\newcommand{\spparend}[1]{{\left(\, #1  \,\right) }}
\newcommand{\angled}[1]{{\left\langle #1  \right\rangle }}
\newcommand{\brackd}[1]{{\left[ #1  \right] }}
\newcommand{\spbrackd}[1]{{\left[\, #1  \,\right] }}
\newcommand{\braced}[1]{{\left\{ #1  \right\} }}
\newcommand{\leftbraced}[1]{{\left\{ #1  \right. }}
\newcommand{\floor}[1]{{\left\lfloor #1\right\rfloor}}
\newcommand{\ceiling}[1]{{\left\lceil #1\right\rceil}}
\newcommand{\barred}[1]{{\left|#1\right|}}
\newcommand{\doublebarred}[1]{{\left|\left|#1\right|\right|}}
\newcommand{\spaced}[1]{{\, #1\, }}
\newcommand{\suchthat}{{\spaced{|}}}
\newcommand{\numof}{{\sharp}}

\newcommand{\half}{{\textstyle\frac{1}{2}}}
\newcommand{\elevenhalves}{{\textstyle\frac{11}{2}}}
\newcommand{\onethird}{{\textstyle\frac{1}{3}}}
\newcommand{\sixteenthirds}{{\textstyle\frac{16}{3}}}
\newcommand{\twentytwothirds}{{\textstyle\frac{22}{3}}}
\newcommand{\onefifth}{{\textstyle\frac{1}{5}}}
\newcommand{\threefifths}{{\textstyle\frac{3}{5}}}
\newcommand{\sixfifths}{{\textstyle\frac{6}{5}}}
\newcommand{\eightfifths}{{\textstyle\frac{8}{5}}}
\newcommand{\sixteenfifths}{{\textstyle\frac{16}{5}}}
\newcommand{\eightteenfifths}{{\textstyle\frac{18}{5}}}
\newcommand{\threetenths}{{\textstyle\frac{3}{10}}}
\newcommand{\twentysixfifteenths}{{\textstyle\frac{26}{15}}}
\newcommand{\fisefiftieths}{{\textstyle\frac{57}{50}}}
\newcommand{\ftwotfifths}{{\textstyle\frac{42}{25}}}
\newcommand{\fotwontwfifths}{{\textstyle\frac{42}{125}}}
\newcommand{\eithontwfifths}{{\textstyle\frac{83}{125}}}

\newcommand{\veps}{{\varepsilon}}
\newcommand{\Sigmastar}{{\Sigma^\ast}}

\newcounter{exnum}[section]
\newenvironment{problem}{{\vskip 0.1in
   \noindent \bf Problem\addtocounter{exnum}{1}~\arabic{exnum}.}}{\vskip 0.1in}

\newtheorem{theorem}{Theorem}
\newtheorem{definition}{Definition}
\newtheorem{corollary}{Corollary}
\newtheorem{lemma}{Lemma}
\newtheorem{fact}{Fact}
\newtheorem{claim}{Claim}

\newenvironment{proof}{{\it Proof:\/}}{$\Box$\vskip 0.1in}

\newcommand{\emparagraph}[1]{{\smallskip\noindent{\em #1\/}}}

\newcommand{\assign}{{\,\gets\,}}



\newcommand{\hwduedate}{{11:59PM, Friday, April~16}} %use if needed, not the real date
\begin{document}

\centerline{\large \bf CS111 Winter'25 ASSIGNMENT 1}
%\centerline{Due date: {on the schedule}

\vskip 0.2in

%%%%%%%%%%%%%%%%%%%%%%%%%%%

\begin{problem}
Give an asymptotic estimate  for the number $D(n)$ of ``D''s printed by Algorithm~\textsc{Print\_Ds} below.
Your solution \emph{must} consist of the following steps:
%
\begin{description} \setlength{\itemsep}{-0.01in}
\item{(a)} First express $D(n)$ using the summation notation $\sum$.
\item{(b)} Next, give a closed-form expression\footnote{A closed-form expression is a formula that can be evaluated
            in some fixed number of arithmetic operations, independent of $n$. For example, $3n^5+n-1$ and $n2^n+5n^2$
            are closed-form expressions, but $\sum_{i=1}^n i^2$ is not, as it involves $n-1$ additions.}
    for $D(n)$. 
\item{(c)}  Finally, give the asymptotic value of $D(n)$ using the $\Theta$-notation.
\end{description}
\noindent
Show your work and include justification for each step. 

\begin{tabbing}
aa \= aa \= aa \= aa \= aa \= aa \= \kill
\textbf{Algorithm} \textsc{Print\_Ds} $(n: \mbox{\bf integer})$ \\
      \> \textbf{for} $i \leftarrow 1$ \textbf{to} $4n$ \textbf{do} \\
      \> \> \textbf{for} $j \leftarrow 3$ \textbf{to} $(2i+1)^2$ \textbf{do} print(``D") \\
      \> \textbf{for} $i \leftarrow 5$ \textbf{to} $n-1$
                         \textbf{do} \\
      \> \> \textbf{for} $j \leftarrow 1$ \textbf{to} $4i$ \textbf{do}  print(``D") 
\end{tabbing}
\smallskip
\noindent
\emph{Note:} If you need any summation formulas for this problem, you are allowed to look them up. You do not need to
prove them, you can just state in the assignment when you use them.
\begin{description} \setlength{\itemsep}{-0.01in}
\item{(a)} In order to express $D(n)$ using summation notation, we first have to analyze how each part of the for
\item loop runs. We notice in the outer for loop, its iteration is from $i=1$ to $i=4n$, which is the range of the
\item summation. Secondly, the inner loop suggests the range for $j=3$ to $(2i+1)^2$
\end{description}
\noindent
\end{problem}

%%%%%%%%%%%%%%%%%%%%%%%%%%%%

\begin{problem} 
(a) Use properties of quadratic functions to prove that $5x^2 \geq (x + 1)^2$ for all real $x \ge 1$.

\smallskip
\noindent 
(b) Use mathematical induction and the inequality from part (a) to prove that $3\cdot 5^n \ge 4^{n+1} + n \cdot 3^n + n^2$ for all integers $n\ge 3$.

\smallskip
\noindent
(c)
Let $g(n) = 4^{n+1} + n \cdot 3^n + n^2$ and $h(n) = 5^n$.
Using the inequality from part~(b), prove that $g(n) = O(h(n))$.
You need to give a rigorous proof derived directly from 
the definition of the $O$-notation, without using any theorems from class.
(First, give a complete statement of the definition. 
Next, show how $g(n) = O(h(n))$ follows from this definition.)
\end{problem}

%%%%%%%%%%%%%%%%%%%%%%%%%%%%

\begin{problem} 
Give asymptotic estimates, using the $\Theta$-notation, for the following functions:
%
\begin{description}\setlength{\itemsep}{-0.01in}
%
\item{(a)} $7n^5 + 5n^3 - 2n^2 + 3$
\item{(b)} $ n^3 {\log^2} n + {n^{2.5}}{\log^5 n}+ 5n^2 {\log_5 n}$
%\item{(c)} $5n^3\log^5 n + 4n^2\sqrt{n} + 3n^3 + 2n^2\sqrt{n}$
\item{(c)} $3 n^4 + n^3 + 2^ {\log n} + n^2 \cdot (0.5)^n $
\item{(d)} $2 n^5 + n^3 \log^4 n + n \cdot (1.5)^n   $
\item{(e)} $n^7 + n^3 \cdot 7^n + n^5 \cdot 4^n$
%
\end{description}
%
Justify your answer in part (a) using the definition of $\Theta$.
Justify your answer in parts (b) - (e) using asymptotic relations between
the basic reference functions: $n^b$, $\log n$, and $c^n$.
\end{problem}


\paragraph{Academic integrity declaration.}
The homework papers must include at the end an academic integrity declaration. This should be a brief paragraph where you state
\emph{in your own words}  (1) whether you did the homework individually or in collaboration with a partner student (if so, provide the name), 
and (2) whether you used any external help or resources. 


%%%%%%%%%%%%%%%%%%%%%%%%%%%%

\vskip 0.1in
\paragraph{Submission.}
To submit the homework, you need to upload the pdf file to Gradescope. If you submit with a partner, you need
to put two names on the assignment and submit it as a group assignment.

\paragraph{Reminders.}
Remember that only {\LaTeX} papers are accepted. 

\end{document}

