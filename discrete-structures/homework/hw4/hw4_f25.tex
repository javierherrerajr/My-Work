\documentclass{article}

\usepackage{fullpage,latexsym,picinpar,amsmath,amsfonts,graphicx}



\setlength{\evensidemargin}{0.1in}
\setlength{\oddsidemargin}{0.1in}
\setlength{\textwidth}{6.6in}
\setlength{\topmargin}{0.0in}
\setlength{\textheight}{8.7in}
\setlength{\headheight}{0in}
\setlength{\headsep}{0in}
\setlength{\topsep}{0in}
\setlength{\itemsep}{0in}
\renewcommand{\baselinestretch}{1.1}
\parskip=0.080in

\newcommand{\parend}[1]{{\left( #1  \right) }}
\newcommand{\spparend}[1]{{\left(\, #1  \,\right) }}
\newcommand{\angled}[1]{{\left\langle #1  \right\rangle }}
\newcommand{\brackd}[1]{{\left[ #1  \right] }}
\newcommand{\spbrackd}[1]{{\left[\, #1  \,\right] }}
\newcommand{\braced}[1]{{\left\{ #1  \right\} }}
\newcommand{\leftbraced}[1]{{\left\{ #1  \right. }}
\newcommand{\floor}[1]{{\left\lfloor #1\right\rfloor}}
\newcommand{\ceiling}[1]{{\left\lceil #1\right\rceil}}
\newcommand{\barred}[1]{{\left|#1\right|}}
\newcommand{\doublebarred}[1]{{\left|\left|#1\right|\right|}}
\newcommand{\spaced}[1]{{\, #1\, }}
\newcommand{\suchthat}{{\spaced{|}}}
\newcommand{\numof}{{\sharp}}

\newcommand{\half}{{\textstyle\frac{1}{2}}}
\newcommand{\elevenhalves}{{\textstyle\frac{11}{2}}}
\newcommand{\onethird}{{\textstyle\frac{1}{3}}}
\newcommand{\sixteenthirds}{{\textstyle\frac{16}{3}}}
\newcommand{\twentytwothirds}{{\textstyle\frac{22}{3}}}
\newcommand{\onefifth}{{\textstyle\frac{1}{5}}}
\newcommand{\threefifths}{{\textstyle\frac{3}{5}}}
\newcommand{\sixfifths}{{\textstyle\frac{6}{5}}}
\newcommand{\eightfifths}{{\textstyle\frac{8}{5}}}
\newcommand{\sixteenfifths}{{\textstyle\frac{16}{5}}}
\newcommand{\eightteenfifths}{{\textstyle\frac{18}{5}}}
\newcommand{\threetenths}{{\textstyle\frac{3}{10}}}
\newcommand{\twentysixfifteenths}{{\textstyle\frac{26}{15}}}
\newcommand{\fisefiftieths}{{\textstyle\frac{57}{50}}}
\newcommand{\ftwotfifths}{{\textstyle\frac{42}{25}}}
\newcommand{\fotwontwfifths}{{\textstyle\frac{42}{125}}}
\newcommand{\eithontwfifths}{{\textstyle\frac{83}{125}}}

\newcommand{\veps}{{\varepsilon}}
\newcommand{\Sigmastar}{{\Sigma^\ast}}

\newcounter{exnum}[section]
\newenvironment{problem}{{\vskip 0.1in
   \noindent \bf Problem\addtocounter{exnum}{1}~\arabic{exnum}.}}{\vskip 0.1in}

\newtheorem{theorem}{Theorem}
\newtheorem{definition}{Definition}
\newtheorem{corollary}{Corollary}
\newtheorem{lemma}{Lemma}
\newtheorem{fact}{Fact}
\newtheorem{claim}{Claim}

\newenvironment{proof}{{\it Proof:\/}}{$\Box$\vskip 0.1in}

\newcommand{\emparagraph}[1]{{\smallskip\noindent{\em #1\/}}}

\newcommand{\assign}{{\,\gets\,}}


\input{pseudocode.tex}
%\newcommand{\hwduedate}{{}}

\begin{document}

\centerline{\large \bf CS111 ASSIGNMENT 4}
%\centerline{due {\hwduedate}}


\vskip 0.15in

%%%%%%%%%%%%%%%%%%%%%%%%%%%%


\newcommand{\calT}{{\mathcal{T}}}



\begin{problem}
Give an asymptotic estimate, using the $\Theta$-notation, of the number of letters printed by the
algorithms given below. Give a complete justification for your answer, by providing an appropriate recurrence
equation and its solution.

\medskip
\noindent
(a) 
\hspace{0.01in}
%
\begin{minipage}[t]{2.4in}
\strut\vspace*{- 2.5 \baselineskip}\newline 
\begin{program}
algorithm |PrintAs|$(n)$
   if $n\le 1$ then
      |print("A")|
   else
      for $j\assign 1$ to $n^2$
         do |print("A")|
      for $i\assign 1$ to $5$ do
         |PrintAs|$(\,\floor{n/2}\,)$
\end{program}
\end{minipage}
%
\hspace{0.4in}
(b) 
\hspace{0.01in}
%
\begin{minipage}[t]{2.4in}
\strut\vspace*{- 2.5 \baselineskip}\newline 
\begin{program}
algorithm |PrintBs|$(n)$
   if $n\ge 4$ then
      for $j\assign 1$ to $n^2$
         do |print("B")|
      for $i\assign 1$ to $6$ do
         |PrintBs|$(\,\floor{n/4}\,)$
      for $i\assign 1$ to $10$ do
         |PrintBs|$(\,\ceiling{n/4}\,)$
\end{program}
\end{minipage}

\medskip
\noindent
(c) 
\hspace{0.01in}
%
\begin{minipage}[t]{2.4in}
\strut\vspace*{- 2.5 \baselineskip}\newline 
\begin{program}
algorithm |PrintCs|$(n)$
   if $n\le 2$ then
      |print("C")|
   else
      for $j\assign 1$ to $n^2$
         do |print("C")|
      |PrintCs|$(\,\floor{n/3}\,)$
      |PrintCs|$(\,\floor{n/3}\,)$
      |PrintCs|$(\,\floor{n/3}\,)$
      |PrintCs|$(\,\floor{n/3}\,)$
\end{program}
\end{minipage}
%
\hspace{0.4in}
(d) 
\hspace{0.01in}
%
\begin{minipage}[t]{2.4in}
\strut\vspace*{- 2.5 \baselineskip}\newline 
\begin{program}
algorithm |PrintDs|$(n)$  
   if $n\ge 5$ then
      |print("D")|
      |print("D")|
     if $(x \equiv 0 \pmod 2)$ then 
         |PrintDs|$(\,\floor{n/5}\,)$
         |PrintDs|$(\,\ceiling{n/5}\,)$
         $x\assign \ x + 3$
      else
         |PrintDs|$(\,\ceiling{n/5}\,)$
         |PrintDs|$(\,\floor{n/5}\,)$
         $x\assign 5x + 3$
\end{program}
\end{minipage}

\noindent
In part~(d), variable $x$ is a global variable initialized to $1$.
\end{problem}


%%%%%%%%%%%%%%%%%%%%%%%%%%%%
%%%%%%%%%%%%%%%%%%%%%%%%%%%%

\vspace{0.1in}
\begin{problem}
A school has three clubs: the Art Club, the Band, and the Computer Science Club, with a total of 129 members across all clubs. The following information is known about the memberships of these clubs:

\noindent 1. The Band has twice as many members as the Art Club, and the Computer Science Club has three times as many members as the Art Club.

\noindent 2. There are 18 members who are in both, the Art Club and the Band, and 20 members who are in both, the Art Club and the Computer Science Club. Additionally, 24 members are in both, the Band and the Computer Science Club.

\noindent 3. There are 11 members who belong to all three clubs.

\noindent Use the inclusion-exclusion principle to determine the number of members in each club. Show your work.
\end{problem}


%%%%%%%%%%%%%%%%%%%%%%%%%%%%

\begin{problem}
A gourmet chocolate shop is preparing custom chocolate boxes, each filled with 68 chocolates selected from four types: Dark Raspberry Night (dark chocolate with raspberries), Hazelnut Noir (dark chocolate with hazelnuts), Espresso Truffle (dark chocolate with coffee cream), and Walnut Maple  (milk chocolate with walnuts and a hint of maple).

\noindent To maintain the perfect balance of flavors, the chocolatier insists on including at least 15 Espresso Truffle  ($e$), but no more than 12 Walnut Maple chocolates ($w$) chocolates. Meanwhile, the number of Dark Raspberry Night ($r$) and Hazelnut Noir ($h$) pieces must each be between 10 and 22.

\noindent How many possible ways can the chocolatier assemble these custom boxes while meeting the flavor requirements?

\noindent You need to give a complete derivation for the final answer, using the method developed in class. 
(Brute force listing of all lists will not be accepted.)

\end{problem}



%%%%%%%%%%%%%%%%%%%%%%%%%%%%

\paragraph{Academic integrity declaration.}
The homework papers must include at the end an academic integrity declaration. This should be a short paragraph where you briefly explain 
\emph{in your own words}  (1) whether you did the homework individually or in collaboration with a partner student (if so, provide the name), 
and (2) whether you used any external help or resources. 

%%%%%%%%%%%%%%%%%%%%%%%%%%%%

\vskip 0.1in
\paragraph{Submission.}
To submit the homework, you need to upload the pdf file to Gradescope. If you submit with a partner, you need
to put two names on the assignment and submit it as a group assignment.
Remember that only {\LaTeX} papers are accepted. 

\end{document}