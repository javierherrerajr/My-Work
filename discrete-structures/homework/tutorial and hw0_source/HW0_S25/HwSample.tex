% https://www.overleaf.com/learn/latex/Main_Page <-- latex resources
% <--- percent sign starts a comment line in Latex

%----------------------------------------------------------------------------------
%
% This is a sample assignment .tex file. Put your name, assignment number and the 
% due date below, as shown.
% If you work with a partner, create another line with command \student{his/her name}
%
% Before you typeset your own assignment try to preview and print this one. If you
% have LaTeX installed on your machine, you need to: 
%   1. Save this in a file, say hw.tex
%   2. Run "pdflatex hw" 
%	3. LaTeX produces a pdf file that you can view with/print any pdf viewer
%
% Alternatively you can also compile and preview it in overleaf.com, the online
% LaTeX editor.
%

%----------------------------------------------------------------------------------

\documentclass[11pt]{article}

%----------------------------------------------------------------------------------
% this is a list of latex packages that may be useful in this document
%----------------------------------------------------------------------------------

\usepackage{fullpage}
\usepackage{graphicx}
\usepackage{amsfonts,amsmath,latexsym,amssymb,amsthm}
\usepackage[nothing]{algorithm}
\usepackage{algorithmicx}
\usepackage[noend]{algpseudocode}
\usepackage{xcolor}

%----------------------------------------------------------------------------------
% this adds macros defined in file macros.tex. don't worry about these.
%----------------------------------------------------------------------------------



\setlength{\evensidemargin}{0.1in}
\setlength{\oddsidemargin}{0.1in}
\setlength{\textwidth}{6.6in}
\setlength{\topmargin}{0.0in}
\setlength{\textheight}{8.7in}
\setlength{\headheight}{0in}
\setlength{\headsep}{0in}
\setlength{\topsep}{0in}
\setlength{\itemsep}{0in}
\renewcommand{\baselinestretch}{1.1}
\parskip=0.080in

\newcommand{\parend}[1]{{\left( #1  \right) }}
\newcommand{\spparend}[1]{{\left(\, #1  \,\right) }}
\newcommand{\angled}[1]{{\left\langle #1  \right\rangle }}
\newcommand{\brackd}[1]{{\left[ #1  \right] }}
\newcommand{\spbrackd}[1]{{\left[\, #1  \,\right] }}
\newcommand{\braced}[1]{{\left\{ #1  \right\} }}
\newcommand{\leftbraced}[1]{{\left\{ #1  \right. }}
\newcommand{\floor}[1]{{\left\lfloor #1\right\rfloor}}
\newcommand{\ceiling}[1]{{\left\lceil #1\right\rceil}}
\newcommand{\barred}[1]{{\left|#1\right|}}
\newcommand{\doublebarred}[1]{{\left|\left|#1\right|\right|}}
\newcommand{\spaced}[1]{{\, #1\, }}
\newcommand{\suchthat}{{\spaced{|}}}
\newcommand{\numof}{{\sharp}}

\newcommand{\half}{{\textstyle\frac{1}{2}}}
\newcommand{\elevenhalves}{{\textstyle\frac{11}{2}}}
\newcommand{\onethird}{{\textstyle\frac{1}{3}}}
\newcommand{\sixteenthirds}{{\textstyle\frac{16}{3}}}
\newcommand{\twentytwothirds}{{\textstyle\frac{22}{3}}}
\newcommand{\onefifth}{{\textstyle\frac{1}{5}}}
\newcommand{\threefifths}{{\textstyle\frac{3}{5}}}
\newcommand{\sixfifths}{{\textstyle\frac{6}{5}}}
\newcommand{\eightfifths}{{\textstyle\frac{8}{5}}}
\newcommand{\sixteenfifths}{{\textstyle\frac{16}{5}}}
\newcommand{\eightteenfifths}{{\textstyle\frac{18}{5}}}
\newcommand{\threetenths}{{\textstyle\frac{3}{10}}}
\newcommand{\twentysixfifteenths}{{\textstyle\frac{26}{15}}}
\newcommand{\fisefiftieths}{{\textstyle\frac{57}{50}}}
\newcommand{\ftwotfifths}{{\textstyle\frac{42}{25}}}
\newcommand{\fotwontwfifths}{{\textstyle\frac{42}{125}}}
\newcommand{\eithontwfifths}{{\textstyle\frac{83}{125}}}

\newcommand{\veps}{{\varepsilon}}
\newcommand{\Sigmastar}{{\Sigma^\ast}}

\newcounter{exnum}[section]
\newenvironment{problem}{{\vskip 0.1in
   \noindent \bf Problem\addtocounter{exnum}{1}~\arabic{exnum}.}}{\vskip 0.1in}

\newtheorem{theorem}{Theorem}
\newtheorem{definition}{Definition}
\newtheorem{corollary}{Corollary}
\newtheorem{lemma}{Lemma}
\newtheorem{fact}{Fact}
\newtheorem{claim}{Claim}

\newenvironment{proof}{{\it Proof:\/}}{$\Box$\vskip 0.1in}

\newcommand{\emparagraph}[1]{{\smallskip\noindent{\em #1\/}}}

\newcommand{\assign}{{\,\gets\,}}



%----------------------------------------------------------------------------------
% the actual source for the document starts here
%----------------------------------------------------------------------------------

\begin{document}

\student{Marek C.\ \ }{86032423} % <-- Replace with your name 
\student{Paco C.\ \ }{86012345} % <-- Cat's name
\vskip 0.1in\noindent\hrule\vskip 0.2in
\assignment{0}                           % <-- ASSIGNMENT NUMBER ******



% ----- start remove ------

{\color{red}

\bigskip
\noindent
What you need to do in this assignment:

\begin{itemize}

\item Put your name and student ID on the top.

\item Revise the academic integrity statement.

\item In Problem 1, implement the following change: Suppose that the first inner loop prints $3i^2$ letters instead of $i^2$.
(Make this change.)
As a result, in the formula for $f(n)$ the first summation will now have coefficient $3$.
Modify this formula and the rest of the calculation to reflect this change.

\item In Problem 2, change the notation, by changing the name of variable $a$ to $b$.

\item Remove these instructions (text in red).

\end{itemize}
}

% ----- end remove -----

%----------------------------------------------------------

\begin{solution}
\noindent(a) 
The inner loop of the first \textbf{for} loop prints $i^2$ letters for each $i = 1,2,...,n+1$. The inner loop of the second \textbf{for} loop prints $2i$ letters for each $i = 1,2,...,n^2$.
Thus denoting $f(n)$ the number of letters ``A" printed we have:
\begin{equation*}
    f(n) = \sum_{i=1}^{n+1} i^2 + \sum_{i=1}^{n^2}(2i) = \sum_{i=1}^{n+1} i^2 + 2\sum_{i=1}^{n^2}i.
\end{equation*}
%
\noindent (b)
Using formulas for the sum of the first $k$ terms of an arithmetic series and the sum of squares of $k$ first integers, we can simplify the above formula as follows:
% 
\begin{align*}
    f(n) &= \textstyle \sum_{i=1}^{n+1} i^2 + 2 \sum_{i=1}^{n^2}i 
    \\
    &= \textstyle \frac{1}{6} (n+1)(n+2)(2n+3) + 2 \cdot \frac{1}{2} n^2(n^2+1)
    \\
    &=\textstyle  \frac{1}{6} ( 2n^3 + 9n^2 + 13n + 6) + (n^4+n^2)
    \\
    &=\textstyle  n^4 + \frac{1}{3} n^3 + \frac{5}{2} n^2+ \frac{13}{6}n + 1
\end{align*}
%
\noindent (c)
We conclude that $f(n) = \Theta(n^4)$, because $f(n)$ is a polynomial of degree $4$.
\end{solution}

%----------------------------------------------------------

\begin{solution}
Use induction to prove the formula for the sum of a geometric sequence (where $a\neq 1$):
\begin{equation*}
\sum_{i=0}^{n} a^i = \frac{a^{n+1}-1}{a-1}
\end{equation*}

\smallskip\noindent
\emph{Base case:} For $n=0$, $\text{LHS} =a^0 = 1$ and $\text{RHS}=\frac{a-1}{a-1} = 1$. So it is true for base case.

\smallskip\noindent
\emph{Inductive Step:} Assume the identity holds for some for $n=k$, that is:
\begin{equation*}
\sum_{i=0}^{k} a^i = \frac{a^{k+1}-1}{a-1}
\end{equation*}
%	
We need to show that it is true for $n=k+1$:
\begin{equation*}
\sum_{i=0}^{k+1} a^i = \frac{a^{k+2}-1}{a-1}
\end{equation*}
		
To do this, starting with the left-hand side, we proceed as follows:
\begin{align*}
			\text{LHS} &= \sum_{i=0}^{k+1} a^i = \sum_{i=0}^{k} a^i + a^{k+1} && \text{(separate last term from the sum)}
			\\
			&= \frac{a^{k+1}-1}{a-1} + a^{k+1}  && \text{(apply inductive assumption)}
			\\
			&= \frac{a^{k+1}-1+ a^{k+1}(a-1)}{a-1} 
			\\
			&= \frac{a\cdot a^{k+1}-1}{a-1} = \frac{a^{k+2}-1}{a-1} = \text{RHS}
\end{align*}
%
Therefore the claim holds for $n=k+1$, completing the inductive step.

From the base case and the inductive step, we can conclude that the identity is true for all $n \ge 0$.
\end{solution}


%----------------------------------------------------------

\vskip 0.2in
\paragraph{Academic integrity declaration.}
This assignment was done jointly by Marek and his dog Paco.

Paco: I did all the work. I did all the math and typed up the solutions. I verified and have a full understanding
of all solutions.

Marek: I was only watching Paco solving these problems, scratching his belly. Afterwards, it took a while, but
he managed to explain to me all solutions. I understand them now and verified that they are indeed correct.

We have not used any external resources.

%----------------------------------------------------------



\end{document}
